%Basics
\documentclass[aps, prl, reprint, a4paper, english, 12pt]{revtex4}
\usepackage[utf8]{inputenc}
\usepackage{babel}

%Symbols and scientifics
\usepackage{amsmath}
\usepackage{commath}
\usepackage{amsfonts}
\usepackage{amssymb}
\usepackage{physics}
\usepackage{mathtools}
\usepackage{siunitx}
\sisetup{
per-mode = fraction ,
round-mode = figures ,
round-precision = 3 ,
scientific-notation = engineering ,
output-decimal-marker = {.} ,
exponent-product = \times ,
separate-uncertainty = true ,
uncertainty-separator = \ ,
output-product = \cdot ,
quotient-mode = fraction ,
range-phrase = - ,
range-units =  single ,
inter-unit-product = \ensuremath{{\cdot{}}} ,
number-unit-product = \ ,
multi-part-units = single ,
}
\usepackage{units}

%Appendix, TOC and Bibliography
\usepackage{appendix}
\renewcommand\appendixtocname{Appendices}
%\usepackage[nottoc]{tocbibind}
\usepackage[lastpage,user]{zref}

%Figures
\usepackage[svgnames]{xcolor} % Required to specify font color
\usepackage{tikz}
\usetikzlibrary{shadings}
\usepackage{float}
\usepackage{rotating}
\usepackage{graphicx}
\usepackage{caption}
\usepackage{wrapfig}
\usepackage[rmargin=2.5cm, tmargin=2.5cm, lmargin=2.5cm, bmargin=2.5cm]{geometry}
\usepackage{xcolor}
\usepackage{etoolbox}
\usepackage{verbatim}
\usepackage[space]{grffile}
\usepackage[final]{pdfpages}
\usepackage{array}
\usepackage{multirow}

%Header footer
\usepackage{fancyhdr}
\pagestyle{fancy}
\lhead{F. G. Kristensen,\\C. V. Sørensen og R. K. F. Wiuff}
\chead{Phonon mechanics in graphene membranes\\}
\rhead{28/9-2017\\Course 34029: Physics Project}
\cfoot{Side \thepage\, af \zpageref{LastPage}}
\renewcommand{\headrulewidth}{0.4pt}
\renewcommand{\footrulewidth}{0.4pt}

%Text tools
\usepackage[normalem]{ulem}
\usepackage{import}
\usepackage{newclude}
\usepackage{url}
\usepackage{lipsum}
\usepackage{microtype}
\usepackage{hyperref}
\hypersetup{
  colorlinks   = true, %Colours links instead of ugly boxes
  urlcolor     = blue, %Colour for external hyperlinks
  linkcolor    = blue, %Colour of internal links
  citecolor   = red %Colour of citations
}
\usepackage[capitalise]{cleveref}
\usepackage{enumitem}
\usepackage{booktabs}
\usepackage{natbib}
\usepackage{silence}
\WarningFilter{revtex4-1}{Repair the float}

%Definitions and new commands
\setlength{\parindent}{0pt}
\setlength{\parskip}{1ex plus 0.5ex minus 0.2ex}
\newcommand{\logas}[1]{\log_{_{10}}{\left( #1 \right)}}
\newcommand{\sins}[1]{\sin{\left( #1 \right)}}
\newcommand{\tans}[1]{\tan{\left( #1 \right)}}
\newcommand{\coss}[1]{\cos{\left( #1 \right)}}
\newcommand{\sinas}[1]{\sin{\left( #1 \degr \right)}}
\newcommand{\tanas}[1]{\tan{\left( #1 \degr\right)}}
\newcommand{\cosas}[1]{\cos{\left( #1 \degr\right)}}
\newcommand{\lnas}[1]{\mathrm{ln}\left( #1 \right)}
\newcommand{\degr}{^{\circ}}
\newcommand{\me}{\mathrm{e}}
\newcommand{\eula}[1]{ \dpd{L }{#1} - \dod{}{t}\left(\dpd{L}{\dot{#1}}\right)}


\begin{document}

%Titlepage herunder:

\title{Phonon mechanics in graphene membranes}
\date{28/9-2017}
\author{Frederik Grunnet Kristensen (s164003)}
\email[E-mail at ]{s164003@student.dtu.dk}
\author{Christoffer Vendelbo Sørensen (s163965)}
\email[E-mail at ]{s163965@student.dtu.dk}
\author{Rasmus Kronborg Finnemann Wiuff (s163977)}
\email[E-mail at ]{s163977@student.dtu.dk}
\affiliation{Technical University of Denmark}
\homepage[Homepage of the Technical University of Denmark ]{http://www.dtu.dk/english}

\begin{abstract}
  \begin{description}
    \item[Background] Since 2010 (Nobel prize). The research on graphene has developed substantially and has spread out in to a wide range of research fields and applications. Because the knowlegde in terms of graphene and especially carbon atoms is well  understood, fairly sophisticated and easy-to-use software has been developed for research and simulation. As graphene is a very versatile material the possibilities for research in a simulation enviroment are virtually limitless. Therefore it is basically possible to make experiments limited only by imagination, in order to discover new properties and possible applications of graphene.
    \item[Purpose] The purpose of the research is to build a foundation for testing in a lab. In that sense the research is fundamental and the work will provide motivation for further practical testing. In more specific terms, the purpose is to simulate phonons in graphene memebranes and find the optimal conditions for producing phonons in the terahertz spectrum.
    \item[Method] The method for building the foundation will be through theoretic work, mostly simulation. Different set-ups will be tested in a simulation enviroment in order to find the optimal conditions for practical implementation, while satisfying the constraints and goals of the set-up.
  \end{description}
\end{abstract}


\maketitle

\pagenumbering{arabic}

\thispagestyle{empty}
\setcounter{page}{1}
\bibliographystyle{unsrtnat}
\bibliography{Bibliography}

\end{document}
