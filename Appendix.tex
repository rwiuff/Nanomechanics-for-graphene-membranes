% !TEX root = Main.tex

\appendix
\appendixpage
\addappheadtotoc
%\section{Animation of \nth{9} mode}
%\begin{center}
%  \animategraphics[autoplay,loop,width=\textwidth]{60}{VNL/Frames/frame}{1}{60}
%\end{center}
\section{Boundary Conditions in Two Dimensions}
When describing the motion of the atoms in the lattice as well as the forces acting upon them, some boundary conditions must be defined. Because the system is very large compared to the characteristic wavelength in the displacement field, periodic boundary conditions (PBC) work well. The actual boundaries as well as the specific normal modes are not of significant importance due the same fact of size difference in the system vs. the characteristic wavelength. \\
An intuitive approach to imagining PBC is to take the plane of the system and fold it in to at torus. All the four edges has now been put together pairwise. This means that any force acting on one atom which causes some form of motion is now translated on to the atom next to it.\\
For an arbitrary system in two dimensions of size $X$ (side lenght) and with atomic spacing $r_{0}$, each axis has $\mathcal{N}=\dfrac{X}{r_{0}}$ atoms. The total amount of atoms in the sheet naturally becomes $N=\mathcal{N}^{2}$. The Bloch wavevector can be described as
\begin{equation}
     \mathbf{q}=(q_{x},q_{y})
\end{equation}
where
\begin{equation}
     (q_{x},q_{y})=\dfrac{2\pi}{X}(m,n)
\end{equation}
Here $m,n$ are integers in the interval $-\dfrac{\mathcal{N}}{2}, +\dfrac{\mathcal{N}}{2}$. The range of $q_{x},q_{y}$ becomes $-\dfrac{\pi}{r_{0}},+\dfrac{\pi}{r_{0}}$.\\
In the specific case of this study we work with a fixed boundary, more specific, a clamped boundary which means that no bending is allowed in the plane. A clamped boundary results in the need for numerical approximations in calculations of the dynamics. As the tools used for calculations are on a"per atom" atom basis, there wont be any need for finite-element models nor any partial differential equations solver (PDE).

\section{Code Compendium}\label{CoCo}
\subsection{NanoSheetCreator.py}
\inputminted[python3=true,bgcolor=Black,linenos=true]{python}{VNL/PythonScripts/Scripts/NanoSheetCreator.py}
\subsection{NanoMembraneCreator.py}
\inputminted[python3=true,bgcolor=Black,linenos=true]{python}{VNL/PythonScripts/Scripts/NanoMembraneCreator.py}
\subsection{00\_FixConstraints.py}
\inputminted[python3=true,bgcolor=Black,linenos=true]{python}{VNL/PythonScripts/Scripts/00_FixConstraints.py}
\subsection{01\_RelaxSheet.py}
\inputminted[python3=true,bgcolor=Black,linenos=true]{python}{VNL/PythonScripts/Scripts/01_RelaxSheet.py}
\subsection{01\_LennardJonesRelax.py}
\inputminted[python3=true,bgcolor=Black,linenos=true]{python}{VNL/PythonScripts/Scripts/01_LennardJonesRelax.py}
\subsection{02\_DynamicalMatrix.py}
\inputminted[python3=true,bgcolor=Black,linenos=true]{python}{VNL/PythonScripts/Scripts/02_DynamicalMatrix.py}
\subsection{03\_SheetVibrations.py}
\inputminted[python3=true,bgcolor=Black,linenos=true]{python}{VNL/PythonScripts/Scripts/03_SheetVibrations.py}
\subsection{DataExtract.py}
\inputminted[python3=true,bgcolor=Black,linenos=true]{python}{VNL/PythonScripts/Scripts/DataExtract.py}
\subsection{MyAnalysisFunctions.py}
\inputminted[python3=true,bgcolor=Black,linenos=true]{python}{VNL/PythonScripts/Scripts/MyAnalysisFunctions.py}
\subsection{ProjectionPlotter.py}
\inputminted[python3=true,bgcolor=Black,linenos=true]{python}{VNL/PythonScripts/Scripts/ProjectionPlotter.py}
\subsection{ZetaPlotter.py}
\inputminted[python3=true,bgcolor=Black,linenos=true]{python}{VNL/PythonScripts/Scripts/ZetaPlotter.py}
\subsection{2DdataExtract.py}
\inputminted[python3=true,bgcolor=Black,linenos=true]{python}{VNL/PythonScripts/Scripts/2DdataExtract.py}
\subsection{2DmodePlotter.py}
\inputminted[python3=true,bgcolor=Black,linenos=true]{python}{VNL/PythonScripts/Scripts/2DmodePlotter.py}