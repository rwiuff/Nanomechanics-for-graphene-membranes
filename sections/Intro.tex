% !TEX root = ../Main.tex

In this paper we will analyse the dynamics and mechanics of carbon nano membranes. To calculate how every atom should be placed and how they react under different circumstances is a long and time consuming task for people doing experimental work. That's why we aim to create, through simulation, comparable data points for further use experimentally.

Since the isolation and characterisation of
Graphene in 2004 by Andre Geim and
Konstantin Novoselov, scientists have marvelled over the physical properties and potential
application of Graphene. Being a relatively new material, many aspects and ideas are being
investigated and researched at all times. Graphene yield extreme tensile strength as well as
extreme electric conductivity, yet its structure is fairly simple. Graphene consists solely of
carbon atoms thus making it, as carbon atoms are greatly understood in terms of chemical
bonding, easy to simulate using specialised software. As graphene is a very versatile
material the possibilities for research in simulation enviroments are virtually limitless.
Therefore it is basically possible to make experiments limited only by imagination, in order
to discover new properties and possible applications of graphene. This saves ressources
before entering the lab, where the simulated reality is tested.

The concept of carbon nano membranes have existed for a couple of years all ready. 

A graphene layer on top of a substrate with different sized and shaped holes, would
form a nanomembrane. It is expected possible to create such a nanomembrane at the
size of few tens of nanometers at Nanotech with Block-copolymer lithography or TEM
structuring of a substrate. In a virtual enviroment, it is possible to simulate phonons
in the graphene atop of these holes in the membrane. The purpose of this project is to
simulate phonons in the nanomembrane and find the optimal conditions for producing
phonons in the terahertz spectrum.

We will employ the software Atomistic ToolKit (ATK) to calculate phonon properties
of membranes as well as performing molecular dynamics of the excited membrane. The
software will enable prompt setup of relevant structures so that more time is free to analysis
and actual simulations.
