\subsection{Clamped System}
This following section focuses on data extraction and analysis for a system of graphene with a free standing hexagon in the middle of a clamped graphene sheet. The free standing hexagon will therefore be equivalent to a hole in the graphene sheet.  
\subsubsection{Frequency vs. hole size}
In order to find the correlation between frequency of the modes in the hole and the hole size, the frequency over each free standing hexagon, which varies in size, is extracted and plotted as a function of the size of the given hole. \\
This is done by defining a sheet of a definite size in VNL and thereafter check the frequencies for the different holes by inserting the different size holes, one after the other. The distance from the edge of the hole, to the edge of the sheet (or the next hole) is called "neck" and has been chosen to be 5 nm for the biggest hole which has a 10 nm diameter. \\
furthermore, only the frequencies for the first 10 modes will be checked. 