% !TEX root = ../Main.tex
\subsection{\faGithub \ Code Repository}
All of the mentioned code and scripts exists as a repository available online:\newline
\url{https://github.com/rwiuff/Nanomechanics-for-graphene-membranes}

\subsection{NanoSheetCreator.py and NanoMembraneCreator.py}\label{NSCS}
The first script generates the graphene membrane with atoms tagged in a hexagonally shaped path. The output is the sheet in the \textit{.hdf5} format.

The workings of the script is briefly described as such:
First the script creates a bravais-lattice consisting of a hexagonal unit cell. Then two carbon atoms are placed in the unitcell and the unit cell is repeated to the users specifications. \cref{2435} demonstrates, by example from NanoSheetCreator.py, how such lattice is created with Nanolanguage.
\onecolumngrid

\begin{listing}[H]
 \inputminted[python3=true,bgcolor=Black,linenos=true,firstline=24,lastline=35]{python}{Listings/NanoSheetCreator.py}
 \caption{Lines 24-35 from the NanoSheetCreator.py shows how Nanolanguage can be used to create a hexagonal bravais lattice}
 \label{2435}
\end{listing}
\twocolumngrid
The user is then asked to input position and size of the hexagonal tag wanted for placement of the hole. Lastly the sheet can be repeated and information about the position and sizes of the tags are printed to a \textit{.txt} file.

\subsection{00\_FixConstraints.py}\label{00}
A historical bug in ATKPython demands consecutive indices for constrained atoms when using the function \textit{phononEigensystem}. Thus a script was written to ensure that tag with these indices were consecutive. Furthermore, when working with two layer membranes, the script tags the individual layers as "Layer1" and "Layer2".
\subsection{01\_RelaxSheet.py and 01\_LennardJonesRelax.py}\label{01}


\subsection{02\_DynamicalMatrix.py}\label{02}
\subsection{03\_SheetVibrations.py}\label{03}
